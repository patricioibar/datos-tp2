\section{Introducción}
En el presente informe se documentan los pasos seguidos para la realización del Trabajo Práctico 2 de la materia Ciencia de Datos (TA047) de la Facultad de Ingeniería de la Universidad de Buenos Aires. El objetivo del trabajo práctico es realizar un análisis exploratorio de conjunto de datos. El análisis incluye la limpieza y preparación de los datos, la realización de consultas y visualizaciones, y la interpretación de los resultados obtenidos. La consigna del trabajo práctico y los datos a analizar pueden encontrarse en el siguiente enlace: \url{https://organizacion-de-datos-7506-argerich.github.io/consigna_tp1_2c2025.html}.

Se trabajó utilizando Python y la librería pandas para la manipulación y análisis de datos. Además, se utilizaron otras librerías como matplotlib y seaborn para la visualización de datos.

Tanto el código fuente, junto con las visualizaciones, como el informe se encuentran disponibles en el siguiente repositorio de GitHub: \url{https://github.com/patricioibar/datos-tp1}